\chapter{Analysis}
\label{cha:analysis}

The IoT technology is improving the ways that we can monitor physical objects. The cheap price of computers, sensors and radio communication chips enables for new usages of these technologies. Within the maritime industry, specifically within the area of container transport monitoring is often done by local means of inspecting the container. The advancements of Bluetooth Low Energy offer us a new way of creating a robust monitoring system, while introducing a web of things paradigm to rather conservative industry. Many containers already include sensors to monitor their state during transport, which forms a question why is not short range radio communication considered to inform nearby technicians about potential problems detected by these sensors? 

\bigskip

Another angle is to consider transmission of the sensed data to the central monitoring system and allow the technician to use their handheld terminal to clear potential alarms within the container and inform the central system about this resolution. In order to be able to do this several questions need to be answered such as what communication paradigm shall the handheld device use to transmit its data? What about the sensor node? Also when we are considering alarms they need to available to the central operator as soon as possible, so how to do this? Also how do we store the data generated from the system?

\bigskip

The literature review (within the scope of the course) and contents of the course provided answers that the system can be built upon. The technologies needed for the deployment of such system already exist, it is primarily question of combining them in such a way that a feasible system can be created. The Bluetooth Low Energy radio since already present on the sensor node and supporting advertisement has provided a way of implementing the proximity alarming system. The handheld is considered something that many technicians will have already, a smartphone. With many smartphones containing Bluetooth Low Energy radio it further enforces the decision to use the technology. Also openness of Android ecosystem allows for simple development, hence the hand terminal device of choice is Android based smartphone.

\bigskip

The communication principles within sending the alarms for storage in centralized server and sending alarm resolution events from a smartphone are natural fit for RESTful communication. Not only it is stateless but also provides opportunities for simple discovery of supported resources and addition of new containers to the system (which is very important in the industry). On the other hand any new alarms generated by the system must be present to the operator at the monitoring station as soon as possible, hence WebSocket protocol is a natural fit to implement this.

\bigskip

The analysis leads to development of a system using several communication principles, where each is suited for their purpose rather than taking a "golden hammer" approach and trying to merge everything using single approach. Also showing the potential of reprogrammable Bluetooth Low Energy beacons for monitoring purposes is a goal in itself.  


\bigskip

Why use Node.js as a technology of choice for development of the sensor node software as well as central server, while there are other environments well suited for this known to the authors of the system? The simple answer is that it provided us with learning opportunity as neither of the authors of the system has worked with Node.js further than development of simple "Hello World" programs or within the scope of the course. Hence the choice cannot really be substantiated beyond this reason. 

\bigskip

Another question is a question of data storage. It is likely that the amounts of data generated in a real system would be rather large, hence usage of a relational database system (these system are deployed nowadays for data warehousing purposes) such as SQLite. This system is good in not only storing the data generated by the system, but also configuration data and metadata regarding the alarms. SQLite was chosen specifically as it did not require running yet another server process on the machine handling the central server part of the system, but also because it was easily available for addition to Node.js pipeline. 

%%% Local Variables:
%%% mode: latex
%%% TeX-master: "../ClassicThesis"
%%% End:
