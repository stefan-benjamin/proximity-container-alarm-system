\chapter{Introduction}
\label{cha:introduction}

The purpose of this report is to describe the design and implementation of maritime container monitoring system. Using different kind of sensors, the system monitors the container conditions and allows the technicians on the ship to detect any potential issues by using a hand-held device. All sensor data is also transmitted to a central server that is used for higher level control and coordination. This approach changes the traditional work flow in witch constant (normally verbal) communication is required between the controller and technicians.

The system is split into three parts:
\begin{enumerate}
\item Central monitoring system - where the personnel has direct overview into what the sensors detect within containers and what alarms are currently active. The central system contains an alarm database, which contains alarm codes with alarm descriptions and possible resolutions. The component is developed on a standard PC or could be possibly created as a cloud based system.

\item Raspberry Pi sensor nodes - Each container has its own controller with sensors attached to the GPIO ports, sensing a predetermined phenomena within the container. These controllers contain logic to raise an alarm in case a sensor or multiple sensors detect values that are out of bounds. This alarms are then transmitted in two ways, to the central monitoring system using a wired connection so the surveillance personnel is informed and via an Bluetooth LE broadcast so passing technicians hand held terminals can pick up the broadcast.

\item Technician hand held terminal - Since the technicians are required to carry out continuous inspections of the containers within the ship to ensure the containers are properly fastened, we propose that a technician would carry a hand terminal (in our case an Android smartphone), which would be able to pick up the Bluetooth LE alarm broadcast as the technician comes in the proximity of the container. The broadcast contains an alarm id which the device then looks up in the central systems database, shows the description of the problem together with a possible resolution.
\end{enumerate}

The system is based on radios available on the Raspberry Pi system, specifically Bluetooth
LE for broadcast of the alarms, and Wi-Fi 802.11 for communication within the central monitoring
system (both systems are expected to be connected on the same network). This is a deviation from existing monitoring systems on ships as they very often require a wired connection (ex: CAN bus).

Both the central monitoring system and the sensing node are built using Node.js. The application for
the Android phone is created in JAVA. The REST principles are used for accessing the sensor
data from raspberry Pi. Wile the central monitoring system needs the capability of rising alarms and dynamically updating the web interface, web sockets are used.

%%% Local Variables:
%%% mode: latex
%%% TeX-master: "../ClassicThesis"
%%% ispell-dictionary: "british" ***
%%% mode:flyspell ***
%%% mode:auto-fill ***
%%% fill-column: 76 ***
%%% End:
