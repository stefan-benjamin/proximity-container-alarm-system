\chapter{Design and Methodology}
\label{cha:design-and-method}

Many of the other natural sciences have labs with equipment that has
to be configured correctly to experimentally test stated hypotheses.
Such experiments must be meticulously planned and designed in advance
to work properly and provide valid and trustworthy results.

As computer scientists, we usually do not work in labs, and our
experiments do not live in petri dishes. Still, we have hypotheses to
test, and thus, experiments to plan. This planning phase is the design
of the experiment, where the authors describe the system intended to
test the hypotheses posed in the introduction. 

It is far easier to reason about the design of a system on a
whiteboard than it is to change what has already been committed to
code, and it is wise to carefully consider all aspects of your design,
before starting to try to build it.

Some hypotheses can be investigated wholly \emph{in vitro}, testing,
\eg one algorithm against another. Other hypotheses require us to
investigate further, involving, \eg potential users or domain experts
to properly evaluate our assumptions. It is therefore crucial to
consider not only \emph{what} you wish to create, but \emph{how} you
propose to evaluate it. For your study to have validity, the design of
the evaluation is every bit as crucial as the design of the object
being evaluated. The credibility of your study also relies on your
ability to communicate the process with which you have reached your
design.

An efficient way of communicating systems design is through the use of
\ac{UML} diagrams, both structurally using class diagrams, and
behaviourally using sequence diagrams, see
\autoref{fig:WebRTCSignalProcess}, both of which can be created
without any external applications.  These diagrams can later be
referred to, or modified in the next chapter, as there may well be
some differences between your design and your implementation.

Indeed, a luxury of this chapter is that the design may well go further
than solely the confirmation or refutation of the hypotheses.  If you
are building a system, this is where you show that you know how to
design one, even if you will actually not be implementing all of it.
If you had sufficient time and resources, \emph{this} is how you would make
your system.

However, before we come to that, it is necessary to investigate
whether the required hypotheses are valid. If they are not, the design
must be reconsidered, and there is only one way to test them, namely
through implementation, and subsequent evaluation.



%% -----------------------------------------------------------------------------------
%% Created using pgf-umlsd
%% http://mirrors.ibiblio.org/CTAN/graphics/pgf/contrib/pgf-umlsd/pgf-umlsd-manual.pdf
%% -----------------------------------------------------------------------------------
\begin{figure}
  \centering\footnotesize\sffamily
  
    \begin{sequencediagram}
       \newthread{p1}{\normalsize peer 1}
       \newinst[1]{SS}{\normalsize signaling server}
       \newinst[1]{p2}{\normalsize peer 2}
       
       \begin{call}{p1}{getUserMedia()}{p1}{return MediaStream}
       \end{call}
       
       \begin{call}{p1}{addStream()}{p1}{}
       \end{call}

       \begin{call}{p1}{createOffer()}{p1}{}
       \end{call}

       \begin{call}{p1}{setLocalDescription()}{p1}{}
       \end{call}

        
       \begin{messcall}{p1}{send offer}{SS}
            \begin{messcall}{SS}{send offer}{p2}
                \begin{call}{p2}{setRemoteDescription()}{p2}{}
                \end{call}
                \begin{call}{p2}{getUserMedia()}{p2}{}
                \end{call}
                \begin{call}{p2}{addstream()}{p2}{}
                \end{call}
                \begin{call}{p2}{setLocalDescription()}{p2}{}
                \end{call} 
                \begin{messcall}{p2}{send answer}{SS}
                \end{messcall}
            \end{messcall}
            \begin{messcall}{SS}{send answer}{p1}
            \end{messcall}
       \end{messcall}  
      \begin{call}{p1}{setRemoteDescription()}{p1}{}
        \end{call}
       \begin{messcall}{p1}{send ICE candidate}{SS}
            \begin{messcall}{SS}{}{p2}
                \begin{call}{p2}{addIceCandidate()}{p2}{}
                \end{call}
                \begin{messcall}{p2}{send ICE candidate}{SS}
                \end{messcall}
            \end{messcall}
            \begin{messcall}{SS}{send ICE candidate}{p1}
            \end{messcall}
       \end{messcall}
       \begin{call}{p1}{addIceCandidate()}{p1}{}
        \end{call}
    \end{sequencediagram}
    \caption{A sequence diagram can communinate the interaction between components efficiently (adapted from \cite{Stephansen2017:2017})}
    \label{fig:WebRTCSignalProcess}
\end{figure}
  
%%% Local Variables:
%%% mode: latex
%%% TeX-master: "../ClassicThesis"
%%% End:
