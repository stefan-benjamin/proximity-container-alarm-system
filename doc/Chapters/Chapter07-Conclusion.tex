\chapter{Conclusion}
\label{cha:conclusion}

This report presented the design and implementation of a maritime container monitoring system. The system consists of 3 main components:
\begin{itemize}
\item Central monitoring system
\item Raspberry Pi 3 sensor nodes
\item Technician hand held terminal
\end{itemize}

These components were integrated in order to address the issue of monitoring the state of the containers not only by displaying information in a centralized manner but also during the visual inspection process performed regularly by the technicians.

While the prototype works as a proof of concept for the system and the technologies used, further development is required in order to achieve a working product. In particular, more sensor information can be retrieved and the central display of status interface can be improved.

When it comes to the technologies, we were able to demonstrate that Node js can be used to program a system of this nature but there is place for improvement on the individual libraries we used. For example, we tried to lower the broadcast transmission power of the BLE module on the Raspberry Pi but we were not able to do this using the Bleno library.

Despite the small encountered limitations, the process used for the project was appropriate and the team is satisfied with the result.  

%%% Local Variables:
%%% mode: latex
%%% TeX-master: "../ClassicThesis"
%%% End:
